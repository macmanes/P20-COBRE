\documentclass[11pt]{article}
\usepackage{framed, color}
\usepackage{textpos}
\usepackage{natbib}
\usepackage[top=1in, bottom=1in, left=.9in, right=.9in]{geometry}
\usepackage{color}
\usepackage{hyperref}
\usepackage{textcomp}
\usepackage{graphicx}
\usepackage{fancybox}
\usepackage{setspace}
\hypersetup{colorlinks=false, urlcolor=blue, citecolor=black}
\usepackage{soul}
\usepackage{geometry}
\usepackage{color}
\newgeometry{top=1in, bottom=1in, left=.75in, right=.75in}
\usepackage{fancyhdr}
\usepackage{wrapfig}
\usepackage{mdframed}
\pagenumbering{arabic}
\usepackage{fontspec}
\setmainfont{Arial}

\linespread{1.1}

\begin{document}


%\parindent 0.000000001in
\setlength{\parindent}{1cm}
\setcounter{page}{0}
\pagenumbering{arabic}



\fancyhead[CO]{Matthew D. MacManes | Specific Aims}
\pagestyle{fancy}
\setcounter{page}{1}
%\noindent \large{\textbf{\textsc{2. Project:}}}
%\normalsize 
%\begin{center}
%\textsc{{i. Significance}} \\
%\end{center}

The maintenance of water balance is critical for survival. Humans are exquisitely sensitive to changes in osmolality, with slight derangement eliciting physiologic compromise. When the loss of water exceeds dietary intake, dehydration - and in extreme cases, death - can occur. Far from uncommon, millions of people die every year as a direct result of dehydration. In contrast to humans, animals living in desert habitats thrive without water and endure extreme heat and intense drought, as a direct result of unique adaptations. These adaptation all them to survive conditions fatal to humans and most other animals. Despite being a well-known ecological phenomenon with obvious implications for human health, we know very little of the underlying mechanisms that allow for survival in desert environments. \textbf{The proposed research uses a novel approach integrating physiology, evolutionary genomics, and computational biology to better understand how animals survive in what appear to be non-survivable conditions.} This proposal represents the foundational steps toward developing the cactus mouse (\textit{Peromyscus eremicus}) as a model system for the study of physiologic water conservation. Indeed, this model offers the scientific community a unique opportunity to gain a deep understanding into the physiology and genomics of osmoregulation in extreme environments – a critically important insight that is impossible using traditional model system like \textit{Mus}, that like humans, die with subjected to these conditions. While not a part of this proposal, this project lays the groundwork for \ul{\emph{my long-term research goal}} – to identify the causal links between phenotype and genotype, using emerging technologies like the CRISPR-Cas9 system. Ultimately, understanding the mechanisms underlying extreme osmoregulation may suggest novel treatment strategies for conditions (e.g. diarrhea) resulting in acute dehydration in humans.\\

\noindent \textbf{SPECIFIC AIM 1:} To characterize the the physiology and adaptive response (differential gene expression, patterns of methylation or isoform use) in desert-adapted mice in response to extreme heat and aridity.  

\begin{quote}
The working hypothesis is that while desert-adapted mice may demonstrate genome wide expression patterns suggestive of stress (e.g. heat shock proteins) during dehydration, these responses function to preserve normal physiology and thus serum electrolytes will be similar to mice with unrestricted access to water. 

\end{quote}

\noindent \textbf{SPECIFIC AIM 2:} To determine the ontogeny of extreme osmoregulatory ability, from the neonatal period during which fluid (milk) intake is obligate through weaning, when oral fluid intake is exceptionally rare. 

\begin{quote}
I hypothesize that patterns of renal gene expression during fetal development through weaning will resemble patterns of gene expression, isoform use, and methylation typical of adult mice when water is freely available. 

\end{quote}

The proposed project aims to integrate studies of physiology, genomics, and computational biology to gain a deep understanding of a fundamental physiological problem – how to conserve water when intake is limited. \ul{\emph{Although dehydration is both common and dangerous, a large swath of the biology underlying its physiological effects is currently invisible to researchers using traditional mammalian models of disease that lack the eco-evolutionary history present in desert-adapted mice}}. This project will fill a critically important gap in our understanding, which is in support of the specific research aims of the National Institute of Diabetes and Digestive and Kidney Diseases (NIDDK).

\newpage
\fancyhead[CO]{Matthew D. MacManes | Research Strategy}
\pagestyle{fancy}
\setcounter{page}{2}
%\noindent \large{\textbf{\textsc{2. Project:}}}
\normalsize 
\begin{center}
\textsc{{i. Significance}} \\
\end{center}

Dehydration, whether caused by exposure to extreme environmental conditions, water deprivation, or by infection (e.g. diarrheal illnesses) represents a significant threat to human life. In spite of modern medicine, millions of people die every year from dehydration. Compounding issues of exposure and illness, are public health issues regarding the delivery of safe drinking water. With global climate change, these challenges are thought to become only more severe and as a result, research providing insight into the mechanisms underlying physiologic resistance to acute dehydration is urgently needed. Unfortunately, similar to humans, the response to acute dehydration in traditionsal mammalian models is generally maladaptive and may include death - this response limits our ability to develop novel insights into this important cause of human mortality. As such, the development of dehydration-tolerant mammalian models will significantly enhance our understanding, and will provide fodder for novel treatments. The proposed work aims study this important human health issue in a uniquely suited novel desert-adapted model organism.
 \\

While the mechanisms underlying physiological compromise in dehydration are well characterized [ref], some animals possess the ability, much unlike humans, to osmoregulate despite extreme heat and a complete lack of extrinsic water intake [ref]. Specifically, highly adapted desert mice may never drink water [ref], produce an extremely viscous urine (or no urine at all) [ref], and excrete urea in the form of uric acid crystals in the feces [ref]. This phenotype results in an animal that is very resistant to dehydration-related physiologic compromise, and is in stark contrast to the phenotype of humans and traditional model organisms (e.g. \textit{Mus} and \textit{Rattus}). Although these organisms are attractive targets for study, they lack the biology that allows for new insight. In contrast with traditional model organisms, non-model desert-adapted organisms may provide a unique opportunity to study dehydration tolerance, though they typically lack many of the genomic and physiologic tools, which inhibits discovery. Despite this limitation, renal gene expression have been characterized for several genes, and was shown to be highly derived in some desert adapted rodents (e.g. \textit{Dipodomys} [ref]), but not in others like (\textit{Notomys}) [ref]. 

 and therefore the extent to which differences in gene expression underlies phenotype remains unknown. In addition to its' generality, both studies of non-model organisms focus on a limited number of genes, and therefore may not appropriately assay the complexity of the genetics of extreme renal osmoregulation. 


In summary, the study of extreme renal osmoregulation is hindered on one hand by a lack of tools, and on the other by a lack of an appropriate phenotype. This limitation may undercut our ability to efficiently develop novel therapies directed at mitigating the untoward effects of dehydration. \\

%The proposed research uses a novel approach integrating physiology, evolutionary genomics, and computational biology to better understand how animals survive in what appear to be non-survivable conditions. Pursuant to this goal, I have developed significant genomic resources in a novel mammalian model, \textit{Peromyscus eremicus} and will apply these tools the understanding of extreme osmoregulation. The contribution is to effectively leverage the power of a sophisticated genomic and physiologic toolset of a model organism against a uniquely adapted rodent, which will allow for a synthetic understanding of extreme osmoregulation, and ultimately novel insights into the cause of - and cure for - dehydration related mortality and morbidity.
\normalsize 
\begin{center}
\textsc{{ii. Innovation}} \\
\end{center}

The proposed work recognizes that successful treatment requires an appropriate model, and while traditional models are wonderful, they lack the biology (extreme osmoregulation) upon which more successful interventions may be modeled. The desert-adapted rodent \textit{P. eremicus} retains many of the beneficial characteristics of model organisms, while enhancing opportunity to assay interesting biological phenomenon. In addition to this fundamental innovation, the project it innovative in a number of other ways.
\begin{itemize}
\item Leverage an unprecedented level of control over the experimental environment by use of a desert chamber where natural conditions can be replicated while simultaneously preserving the ability to manipulate water availability. 
\item Link detailed information on physiology and metabolism to genomic information using a novel analytical tools under active development in the lab.
\end{itemize}

 

\newpage

\linespread{1.2}

\noindent \textbf{Aim 1:} \ul{To characterize the the physiology and adaptive response (differential gene expression, patterns of methylation or isoform use) in desert-adapted mice in response to extreme heat and aridity.} I hypothesize that, as a result of unique mechanisms related to solute and water balance, that average serum electrolyte concentrations will remain relatively constant throughout various experimental manipulations, but the variance in measured levels between individuals will increase in the most extreme conditions. These differences will be echoed in differences in urine electrolytes and concentration. \\

To better understand the physiological effects of desert conditions on rodents, I will relate multiple physiological variables to differences in temperature, relative humidity, and water availability. These experiments are fundamentally linked to a series of environmental manipulations, described in \hyperlink{Figure 1}{Figure 1}. The experimental design is fully factorial -- meaning that the focal experimental parameter (e.g. temperature) will be tested in the context of the full range of other conditions (e.g. humidity, water availability). Animal care is standardized between experiments and includes measures to reduce the water content of food and bedding materials. Both of these will be dried in a standard desiccation oven to less than 1\% water/volume. Twenty individuals per treatment will be included -- power analyses suggest this sample size will allow me to garner statistical support for patterns with small to medium effect sizes. Together, this design will allow me to tease-apart the physiologic and genomic response to the various conditions. \\

\vspace{2mm}

\begin{mdframed}
 \begin{center}
  \includegraphics[width=1\textwidth]{exp_design_fig.eps}
 \end{center} 

\noindent \small{Figure 1: Animals are relegated into either hot or cold treatments. Within treatments (n=20 per treatment), animals are exposed to two weeks of varying levels of aridity, from simulated rainfall where water is available \textit{ad libitum}, to dry, where no water is available. RH=relative Humidity}

\end{mdframed}

\vspace{5mm}


For each experiment in Aim 1, I will collect information including metabolism, urine concentration and electrolyte content, serum electrolyte content, as well as more basic measures like body temperature and body weight. In the context of limited water intake, how animals achieve electrolyte balance is unknown. Electrolytes are both easy to assay and are critical to physiological well being. Indeed, proper electrolyte balance is fundamental to all other physiological processes like neuronal signal transduction and muscle (including cardiac) contractility. Here, electrolytes will be measured using the VetScan VS2 critical care panel which includes ALT, BUN, Cl, CRE, GLU, K, Na, bicarbonate ion in a 100uL sample volume. \\

In addition to assaying electrolytes themselves, I will conduct measures of urine electrolytes and specific gravity, as the urinary system represents that major pathway through which these chemicals are lost. These parameters will be measured using an Atago UG-$\alpha$ urine refractometer and tests conducted at the IDEXX reference lab. Lastly, I will weigh animals to the nearest 0.1gm every other day, including the day of sacrifice. Body temperature will be assayed with weighing using a digital thermometer and probe designed by World Precision Instruments (Sarasota, FL). In connection with this, feces will be collected and water content will be assayed using standard methods. \\

I will collect key metabolic parameters such as carbon dioxide production and oxygen consumption that may influence water consumption. These tests will be measured during a twenty four hour period at the end of the experimental manipulation, just prior to euthanasia, using a metabolic chamber (Sable Inc.) modified for use in the desert chamber. Together, these data will represent a uniquely rich characterization of the physiological state of a desert rodent held in captivity but importantly, exposed to conditions typical of the natural environment. \\


 \begin{center}
  \includegraphics[width=1\textwidth]{Aim1-table1.eps}
 \end{center} 

\noindent \small{Table 1: Say something about predictions here.}

\vspace{5mm}


\noindent \textbf{Aim 1a-1:} Determine the physiologic response to drinking-water deprivation in the desert-adapted rodent \textit{P. eremicus}. \\


Background: The human body consist of 55\% water. Far from static, proper physiologic function requires water for countless processes including signal transduction, pH balance, and the removal of metabolic waste. To accomplish these things, approximately 2 liters of fluid are used daily. These losses are accelerated greatly in extremes of heat and aridity. Though the body possesses limited reserves, when loss exceeds intake over a period of time, dehydration and in extreme cases, death can occur.   


Water availability is thought to be one of the key attributes that determines the hostility of a given environment. For instance, ecosystems like deserts present unique challenges to the animals that inhabit them -- the ways in which animals overcome these challenges may allow researchers unique insights with biomedical importance. Importantly, while we have an understanding of how dehydration looks in the context of illness in non-desert-adapted animals, we have no idea what this looks like in healthy desert-adapted animals. \\

Research Plan:To accomplish this aim, I will analyze physiologic and genomic data from animals held with and without drinking water, factorial with respect to the other conditions. The specific experiments and the types of data collected are described above. Patterns of electrolyte levels and urine concentration that are related to this contrast (wet versus dry) will allow me to understand the how water deprivation may be related to water.  \\

Outcome: These data will be compiled into a large data matrix, with statistical analyses focusing on the identifying patterns related to differences in water availability. In addition, because behavior related to desert survival will be characterized as part of a separate project, changes in electrolytes will be linked to individual level differences in behavior. Ultimately, these physiology data will be combined with the genomic data to understand the relationship between physiology and genomics. \\

Preliminary data: I have characterized the electrolyte profile of 2 individuals housed at 70F, 50\% RH, water \textit{ad lib} and two individuals housed in identical conditions except that drinking water was withheld. Despite being housed in typical laboratory conditions, these animals have remarkably unusual electrolyte panel. For instance, mean serum potassium in an un-hemolyzed sample is unusually high at 8.1mg/dL, while Creatinine is low, with a mean measurement of 0.25mg/dL. mean blood urea nitrogen (BUN) is 47mg/dL. In contrast, animals without \textit{ad lib} water were essentially... \\  

\noindent \textbf{Aim 1a-2:} Determine the physiologic response to extreme temperature in the desert-adapted rodent \textit{P. eremicus}. \\

Background: While water stress is obviously important to the survival of desert rodents - a phenotype which is relevant to human health and wellness, extreme temperatures represent another way in which physiological processes may be challenged. While desert animals may thrive in extreme heat, humans cannot. Understanding the physiological response is characterized in humans, but not in other animals evolved in these conditions. Genes like the heat-shock proteins are protective in humans, but no record of their activity on desert rodents is known. \\

Research Plan: 

Outcome:

\noindent \textbf{Aim 1a-3:} Determine the physiologic response to relative humidity in the desert-adapted rodent \textit{P. eremicus}. \\ 

\begin{itemize}
\item Describe role of serum electrolytes.
\item How are these things regulated in the kidney (link to genes)
\item typical dehydration panel
\item What I'm going to measure
\item expected outcome
\end{itemize}



\noindent \textbf{Aim 1b:} \ul{Define patterns of gene expression | isoform use | methylation given differences environmental condition}. {I will understand the genetic response to extreme heat and aridity via a series of bisulfite sequencing and RNAseq experiments, and will link these patterns to individual physiologic state as defined in Aim 1.} I hypothesize that genes responsible for water and solute transport will be particularly active in the most extreme conditions.\\

Background: The genomic processes related to desert survival have yet to be characterized. The few studies of genetics that have been done have focused on the role of single members of the Aquaporin gene family (but see \cite{Bartolo:2007hy}), which are large membrane-bound proteins that are critically involved in renal water transport \citep{Kwon:2009bv,Verkman:2002ww,Brown:1995vo,Nielsen:1995cb}. These studies have shown that changes in Aquaporin (AQP) protein abundance and expression may be related to water availability \citep{Boselt:2009fb, Gallardo:2005fm,Bozinovic:2003eg}. In addition to changes in expression, another study showed that the AQP4 pathway was completely lost in the desert rodent \textit{Dipodomys merriami merriami} \citep{Huang:2001ti}. Despite these studies, we have a limited understanding of the genomics of renal water and solute regulation in desert animals. While AQPs are functionally important, water and solute balance is extraordinarily complex, and therefore single-gene studies are necessarily limited in their purview. A more complete understanding of this phenotype and its mechanistic underpinnings will require a sophisticated genome-level approach, which will be the outcome of the proposed research. 

Research Plan: For analysis of the bisulfite sequence data, an accurate genome assembly is required. Using the existing draft genome (sequenced using startup funds) as a starting point, I will complete the genome assembly of the primary study organism, \textit{Peromyscus eremicus}. This process will be significantly aided by leveraging the existing genome sequence data available from several other \textit{Peromyscus} species. Specifically, an additional 60X coverage Illumina dataset using 150bp paired end and mate-pair sequencing will supplement the current 40X dataset. Simultaneous with this, I will sequence up to 10X coverage using long-read PacBio sequencing. A short-read assembly of Illumina data will be done using AllPaths-LG \citep{Maccallum:2009du}. Illumina contigs will be assembled with the corrected PacBio data using the software package SGA \citep{Simpson:2012ef} or an overlap-layout-consensus assembler (\textit{e.g.} Celera, \cite{Miller:2008jx}) to generate a primary \textit{de novo} genome assembly. Assemblies will be validated by use of the CEGMA \citep{Parra:2007df} and REAPR \citep{Hunt:2013hj} algorithms. In general, this approach has been shown to be successful in producing high quality draft assemblies, though multiple assembly methods will be evaluated as per the findings of my previous work \citep{Bradnam:2013gx}.\\

To annotate the genome, RNA and smRNA sequence data from several tissues, including liver, kidney, and brain will be generated. Sequence data will consist of Illumina strand-specific small-insert 150bp paired end sequencing. Each tissue type will be sequenced to a depth of 100 million reads. Assembly will be accomplished using the Trinity software package \citep{Haas:2013jq,Grabherr:2011jb} after error correction \citep{MacManes:2013ec} and quality trimming \citep{MacManes:2013ex}. Annotation will be completed using the programs Trinotate (\url{http://trinotate.sourceforge.net/}) and Maker \citep{Cantarel:2008jo}. \\

To identify patterns of differential isoform use, I will generate normalized cDNA libraries using X, and sequence using the single molecule Pacific Biosystems technology. This approach offers significant advantage over short read high-throughput sequencing technologies (e.g. Illumina, Ion Torrent) where accurate assembly of isoforms is extremely problematic \citep{Pyrkosz:2013tm}.


In addition to the assembly and annotation of the \textit{P. eremicus} genome, a secondary result of this work is methods development (\textbf{enhancing infrastructure}). To this end, I have already already released a transcriptome assembly pipeline (\url{http://sourceforge.net/projects/tamrs/}) and automated quality control software (\url{http://sourceforge.net/projects/qcpro/}). In addition to this, I am an active developer of the transcriptome assembly program Trinity and annotation software Trinotate. Given the popularity of high-throughput sequencing, the demand for these types of tool development programs will likely increase. \\

Outcome: This work will allow me to look deeply into the genomic processes that underlie adaptation. 

Preliminary Data: To date, I have generated a RNAseq dataset that consists of approximately 30M 150nt SE Illumina reads from the same 5 animals housed in the 'cold/simulated rain' treatment group from which I collected physiology data


\begin{itemize}
\item Describe typical renal gene expression profile
\item link to electrolytes
\item what I'm going to measure and how
\item BiSeq, RNAseq, PacBio transcriptome to get at isoforms
\item expected outcome
\end{itemize}



\noindent \textbf{Aim 1c:} Correlate physiology (=phenotype) with genomic patterns. \\

\begin{itemize}
\item Diff gene expression
\item coexpression networks
\item Patterns of isoform use
\item diff methylation
\end{itemize}





\noindent \textbf{Aim 2:} \ul{Given the transition from the obligate intake of fluids as infants, to it’s complete absence later in life, the ontogeny of physiologic water conservation will be elucidated.} \\

Given that desert adapted mice, capable of surviving without water are as neonates dependent on liquid intake, \ul{the study of the ontogeny of physiologic water conservation is extremely interesting and relevant to the current work.} This phenomenon will be explored using neonate mice in the treatments listed in Aim 1, and methods described in Aim 2. Five neonate mice will be culled per treatment at 3 different timepoints (immediately after birth, mid-lactation, 1 day after weaning). I hypothesize that patterns of gene expression and methylation will resemble those common in conditions where water is available \textit{ab lib}. \\












%\newpage
%
%\begin{center}
%\textsc{{ii. Background}} \\
%\end{center}
%
%The study of adaptation, or the process through which animals become fitted to their environment has intrigued researchers for decades \citep{Darwin:1859tm, Fisher:1930wy}, though only recently have we had the ability to study the underlying genomic mechanisms. Interestingly, researchers interested in understanding the genetics of adaptation have the ability to ground modern studies of genetics on decades of work aimed at understanding the ecological context within which adaptation occurs. One particularly salient example of the connection between studies of ecology and natural history and modern genomics can be found in the study of physiologic adaptation to desert conditions. Here, remarkable physiologic, morphologic \citep{Dickinson:2007jn,Huntley:1984us,SchmidtNielsen:1950wg,SchmidtNielsen:1952wp} and behavioral \citep{NAGY:1994vd} adaptation has been studied in the context of desert ecology. These studies provide a rich context for the current work, which aims to define the relationship between physiology and genomics in rodents able to thrive in amongst the most harsh of conditions on Earth. 
%
%Though classic research relating morphology (especially renal ultrastructure) and physiology to desert adaptation has been done, we know virtually nothing about how extreme heat and aridity may affect other core physiological and metabolic processes. For instance, the maintenance of normal serum electrolyte concentration is challenging in the context of desert conditions. Indeed, electrolyte derangement is often the ultimate cause dehydration-related death, yet for the vast majority of desert animals, we know nothing of electrolyte balance - not even typical values! Understanding these critical processes is fundamental to our understanding of physiological adaptation, and will be accomplished as part of this project. 
%
%Lastly, 
%
%
%\textbf{Specific Aim 1:}  
%
%\textbf{Specific Aim 2:} 
%
%\textsc{{Preliminary Data:}} . 
%
%\begin{center}
%\textsc{{iv. Broad Impacts}} \\
%\end{center}
%How desert animals survive without water is an intrinsically interesting example of extreme physiology and adaptation. Indeed, telling people of the remarkable story of a rodent that may never drink water is immediately captivating. Having told hundreds of people, including children, adults, scientifically literate and not, the most common response is something like "Wow! That is amazing. How do they do that?" I intend to \textbf{broaden dissemination} by capitalizing on this common response, using traditional and non-traditional methods. While lecture has been the mainstay of scientific communication, its audience tends towards the educated/affluent subset of the population. Given one of my primary goals as a scientists is to broadly share my work, I will use non-traditional modes of communication which will include social media and a blog. All publications are hosted on a public preprint server, and published using using open access options. All data and code is freely available using public repositories like Github, Dryad, or Figshare. In addition to this, I will develop an outreach program that aims to present research findings to school-aged children. These program will be multi-dimensional, with programs geared towards several different age groups. 
%
%\textbf{Broadening participation} is an issue about which I care deeply. As a scientist of Native American descent, I strive to be a role model and mentor for underrepresented groups. To this end, I am developing an internship program- in collaboration with the American Indian Higher Education Consortium (\url{http://aihec.org/}), where qualified Native students may spend a summer working in my lab either in directed or independent research programs. It is my hope that students introduced to my lab via internship may ultimately elect to join the lab. This method of recruitment supplements my general interest in recruiting students from underrepresented groups. 
%
%Lastly, my efforts aimed at broadening dissemination and participation are all within the context of a research program with substantial \textbf{benefits to society}. First, Earth is becoming both warmer and drier, and accurate prediction of animals' response to a changing climate requires knowledge of how animals currently living in dry/hot climates survive - this project will deepen the requisite knowledge. Next, kidney disease effects millions of Americans (\url{http://www.cdc.gov/diabetes/projects/pdfs/ckd_factsheet.pdf}). Though the causes are diverse, the pathophysiology of many resemble dehydration (\textit{e.g.} low renal perfusion pressure). Thus, having a better understanding of how desert rodents endure water stress seemingly without complication may enhance our ability to explain why humans and most other mammals cannot.       
%
%\begin{center}
%\textsc{{v. Summary \& Future Directions}} \\
%\end{center}
%
%The maintenance of water balance in animals is one of the most important physiologic processes, and is critical to survival. Osmoregulation in animals living in desert environments is particularly challenging, as extreme heat and aridity is common. Gaining a deeper understanding of the physiological adaptations that allow for desert survival is important and has obvious implications for climate change science, conservation, and human health and medicine. The proposed research aims to generate a uniquely rich dataset, leveraging cutting edge genomic techniques against careful characterization of physiology, all within an ecological context of desert life. A carefully constructed plan for broadening dissemination and participation ensures that the scientific process as well as its results will be available to all interested parties.  
%
%This proposal represents the foundational steps toward developing \textit{P. eremicus} as a model system for the study of desert adaptation. Indeed, this model offers the scientific community a unique opportunity to gain a deep understanding into the physiology and genomics of desert adaptation- an insight which is impossible to achieve using traditional model system. While not a part of this proposal, this work lays the groundwork for future studies where causal links are made between phenotype and genotype, using technologies like CRISPR-Cas9 and RNAi. 




%\textsc{{Preliminary Data:}} To date, I have generated 40X coverage using a small-insert Illumina dataset and a 3X size selected PacBio dataset using P5-C3 chemistry. I have assembled these reads using a novel approach (in brief, SGA assembly of Illumina data, PacBio reads 'spiked-in' at fm-merge stage). This preliminary assembly strategy has shown promise. A relatively unoptimized process has assembled approximately 2.9Gb of the estimated 3.0Gb genome with a scaffold N50 of 10kb. While still relatively fragmented, we are approaching our short term assembly goal of N50 $\geq$ 25kb.  \\









\newpage
\setcounter{page}{1}
%\thispagestyle{empty}
\singlespacing
\bibliographystyle{model2-names.bst}
\bibliography{formatted.bib}


































\end{document}
