\documentclass[11pt]{article}
\usepackage{framed, color}
\usepackage{textpos}
\usepackage{natbib}
\usepackage[top=1in, bottom=1in, left=.9in, right=.9in]{geometry}
\usepackage{color}
\usepackage{hyperref}
\usepackage{textcomp}
\usepackage{graphicx}
\usepackage{fancybox}
\usepackage{setspace}
\hypersetup{colorlinks=false, urlcolor=blue, citecolor=black}
\usepackage{soul}
\usepackage{geometry}
\usepackage{color}
\newgeometry{top=1in, bottom=1in, left=.75in, right=.75in}
\usepackage{fancyhdr}
\usepackage{wrapfig}
\usepackage{mdframed}
\pagenumbering{arabic}
\usepackage{fontspec}
\setmainfont{Arial}

\linespread{1.1}

\begin{document}


%\parindent 0.000000001in
\setlength{\parindent}{1cm}
\setcounter{page}{0}
\pagenumbering{arabic}



\fancyhead[CO]{Matthew D. MacManes | Specific Aims}
\pagestyle{fancy}
\setcounter{page}{1}
%\noindent \large{\textbf{\textsc{2. Project:}}}
%\normalsize 
%\begin{center}
%\textsc{{i. Significance}} \\
%\end{center}

The maintenance of water balance is critical for survival. Humans are exquisitely sensitive to changes in osmolality, with slight derangement eliciting physiologic compromise. When the loss of water exceeds dietary intake, dehydration - and in extreme cases, death - can occur. Far from uncommon, millions of people die every year as a direct result of dehydration. In contrast to humans, animals living in desert habitats thrive without water and endure extreme heat and intense drought, as a direct result of unique adaptations. These adaptation all them to survive conditions fatal to humans and most other animals. Despite being a well-known ecological phenomenon with obvious implications for human health, we know very little of the underlying mechanisms that allow for survival in desert environments. \textbf{The proposed research uses a novel approach integrating physiology, evolutionary genomics, and computational biology to better understand how animals survive in what appear to be non-survivable conditions.} This proposal represents the foundational steps toward developing the cactus mouse (\textit{Peromyscus eremicus}) as a model system for the study of physiologic water conservation. Indeed, this model offers the scientific community a unique opportunity to gain a deep understanding into the physiology and genomics of osmoregulation in extreme environments – a critically important insight that is impossible using traditional model system like \textit{Mus}, that like humans, die with subjected to these conditions. While not a part of this proposal, this project lays the groundwork for \ul{\emph{my long-term research goal}} – to identify the causal links between phenotype and genotype, using emerging technologies like the CRISPR-Cas9 system. Ultimately, understanding the mechanisms underlying extreme osmoregulation may suggest novel treatment strategies for conditions (e.g. diarrhea) resulting in acute dehydration in humans.\\

\noindent \textbf{SPECIFIC AIM 1:} To characterize the the physiology and adaptive response (differential gene expression, patterns of methylation or isoform use) in desert-adapted mice in response to extreme heat and aridity.  

\begin{quote}
The working hypothesis is that while desert-adapted mice may demonstrate genome wide expression patterns suggestive of stress (e.g. heat shock proteins) during dehydration, these responses function to preserve normal physiology and thus serum electrolytes will be similar to mice with unrestricted access to water. 

\end{quote}

\noindent \textbf{SPECIFIC AIM 2:} To determine the ontogeny of extreme osmoregulatory ability, from the neonatal period during which fluid (milk) intake is obligate through weaning, when oral fluid intake is exceptionally rare. 

\begin{quote}
I hypothesize that patterns of renal gene expression during fetal development through weaning will resemble patterns of gene expression, isoform use, and methylation typical of adult mice when water is freely available. 

\end{quote}

The proposed project aims to integrate studies of physiology, genomics, and computational biology to gain a deep understanding of a fundamental physiological problem – how to conserve water when intake is limited. \ul{\emph{Although dehydration is both common and dangerous, a large swath of the biology underlying its physiological effects is currently invisible to researchers using traditional mammalian models of disease that lack the eco-evolutionary history present in desert-adapted mice}}. This project will fill a critically important gap in our understanding, which is in support of the specific research aims of the National Institute of Diabetes and Digestive and Kidney Diseases (NIDDK).

\newpage
\fancyhead[CO]{Matthew D. MacManes | Research Strategy}
\pagestyle{fancy}
\setcounter{page}{2}
%\noindent \large{\textbf{\textsc{2. Project:}}}
\normalsize 
\begin{center}
\textsc{{i. Significance}} \\
\end{center}

Dehydration, whether caused by exposure to extreme environmental conditions, water deprivation, or by infection (e.g. diarrheal illnesses) represents a significant threat to human life. In spite of modern medicine, millions of people die every year from dehydration. Compounding issues of exposure and illness, are public health issues regarding the delivery of safe drinking water. With global climate change, these challenges are thought to become only more severe and as a result, \ul{research providing insight into the mechanisms underlying physiologic resistance to acute dehydration is urgently needed.} The response to acute dehydration in humans and traditional mammalian models is generally maladaptive and may include death - this response limits our ability to develop novel insights into this important cause of human mortality. As such, the study of dehydration-tolerant mammalian models will significantly enhance our understanding, and will provide fodder for novel treatments. \textbf{The proposed work aims study extreme osmoregulation in a uniquely suited novel desert-adapted model organism.}

While the mechanisms underlying physiological compromise in dehydration are well characterized (Roberts, Pope, Newson, Lolait, \& O’Carroll, 2010), some animals possess the ability, much unlike humans, to osmoregulate despite extreme heat and a complete lack of extrinsic water intake (Nagy & Gruchacz, 1994). Specifically, highly adapted desert mice may never drink water, produce an extremely viscous urine, or no urine at all, and excrete urea in the form of uric acid crystals in the feces (K. Schmidt-Nielsen & Schmidt-Nielsen, 1952). This phenotype results in an animal that is very resistant to dehydration-related physiologic compromise, and is in stark contrast to the phenotype of humans and traditional model organisms (e.g. \textit{Mus} and \textit{Rattus}). Although model organisms are attractive targets for study, they lack the requisite biology which may limit insight. In contrast with traditional model organisms, non-model desert-adapted organisms may provide a unique opportunity to study dehydration tolerance, though they typically lack many of the genomic and physiologic tools characteristic of model organisms. Despite this, renal gene expression has been characterized for several genes in desert animals, and was shown to be highly derived in some (e.g. \textit{Dipodomys} (Huang et al., 2001)), but not in others (e.g. \textit{Notomys} (Weaver, Walker, Alcorn, & Skinner, 1994)). No studies characterizing genome-wide patterns of gene expression, methylation or isoform in desert-adapted water stressed animals has been done and therefore the extent to which differences in these parameters underlie phenotype remains unknown. \ul{The proposed work effectively integrates the power of a model organism with the unique biology of a desert-adapted rodent, the cactus mouse (\textit{Peromyscus eremicus}), to generate insights into extreme osmoregulation not current possible.}
%\vspace{-3mm}
%The proposed research uses a novel approach integrating physiology, evolutionary genomics, and computational biology to better understand how animals survive in what appear to be non-survivable conditions. Pursuant to this goal, I have developed significant genomic resources in a novel mammalian model, \textit{Peromyscus eremicus} and will apply these tools the understanding of extreme osmoregulation. The contribution is to effectively leverage the power of a sophisticated genomic and physiologic toolset of a model organism against a uniquely adapted rodent, which will allow for a synthetic understanding of extreme osmoregulation, and ultimately novel insights into the cause of - and cure for - dehydration related mortality and morbidity.
\normalsize 
\begin{center}
\textsc{{ii. Innovation}} \\
\end{center}

The proposed work recognizes that successful treatment requires an appropriate model, and while traditional models are powerful, they lack the biology (extreme osmoregulation) upon which more successful interventions may be modeled. The desert-adapted rodent \textit{P. eremicus} retains many of the beneficial characteristics of model organisms, while enhancing opportunity to assay interesting biological phenomenon. In addition to this fundamental innovation, the project it innovative in a number of other ways.
\begin{itemize}
\item Experimental, conceptual, theoretical, and technical innovation: The proposed project leverages unprecedented control over environmental conditions using an ideally suited novel model organism and unique analytical methods to understand basic physiology in dehydration resistant organisms.

\end{itemize}

 

\newpage

\linespread{1.2}

The proposed project fits well in line with the general and specific goals of the National Institute for Digestive Disorders and Kidney (NIDDK). 




\newpage
\setcounter{page}{1}
%\thispagestyle{empty}
\singlespacing
\bibliographystyle{model2-names.bst}



\end{document}

\bibliography{}
